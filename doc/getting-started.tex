\documentclass[12pt]{report}
% \usepackage{array}
\usepackage{ug_contact}
% \usepackage{amsfonts,amsmath,multirow,enumitem}
\usepackage{alltt}
\usepackage[colorlinks=true, linkcolor=blue, citecolor=blue, linktoc=all]{hyperref}
\usepackage{bookmark}
\usepackage{step}\usepackage{newtxmath}\usepackage[T1]{fontenc}

\usepackage{hugepag,epsfig,ignore,rotating}
\usepackage{amsfonts,amsmath}
\usepackage{color}

\definecolor{darkblue}{rgb}{ 0.04, 0.04, 0.4}
\definecolor{darkgreen}{rgb}{0.05, 0.50, 0.05}
\definecolor{premium}{rgb}{ 0.04, 0.04, 0.6}
\definecolor{magenta}{rgb}{ 0.6, 0.0, 0.6}
\newcommand{\blue}[1]{{\color{blue}#1}}
\newcommand{\red}[1]{{\color{red}#1}}
\newcommand{\magenta}[1]{{\color{magenta}#1}}
\newcommand{\premium}[1]{{\color{premium}#1}}

\newcommand{\bsea}{\begin{subequations}\begin{eqnarray}}
\newcommand{\esea}{\end{eqnarray}\end{subequations}}
\newcommand{\bseal}[1]{\begin{subequations}\label{#1}\begin{eqnarray}}
\newcommand{\eseal}{\end{eqnarray}\end{subequations}}

\def\Z{\mathbb{Z}}
\def\R{\mathbb{R}}
% \def\R{\hbox{I\kern-.2em\hbox{R}}}
\renewcommand{\vec}[1]{ \mathbf{#1} }
\newcommand{\mat}[1]{ \mathbf{#1} }
% \newcommand{\transp}{^{\mathrm{\scriptscriptstyle T}}}
% \newcommand{\grid}[1]{{\cal #1}}

\newcommand{\fpd}[2]{ \frac{\partial #1}{\partial #2} }	
\newcommand{\spd}[2]{ \frac{\partial^2 #1}{\partial #2^2} }
\newcommand{\mpd}[3]{ \frac{\partial^2 #1}{\partial #2 \partial #3} }
\newcommand{\fod}[2]{ \frac{{\rm d} #1}{{\rm d} #2} }
\newcommand{\sod}[2]{ \frac{d^2 #1}{d #2^2} }	
\newcommand{\mtd}[1]{ \frac{{\rm D} #1}{{\rm D}t} }
\newcommand{\txtfpd}[2]{ \partial #1/\partial #2 } 
\newcommand{\txtspd}[2]{ \partial^2 #1/\partial #2^2 } 
\newcommand{\txtmpd}[3]{ \partial^2 #1 / \partial #2 \partial #3 }
\newcommand{\txtfod}[2]{d #1/d #2}	
\newcommand{\txtmtd}[1]{ {\rm D} #1/{\rm D}t }

\newcommand{\acos}{ \mathrm{acos} }	
\newcommand{\asin}{ \mathrm{asin} }	
\newcommand{\atan}{ \mathrm{atan} }	

\newcommand{\bunit}[1]{[\mathrm{#1}]}
\newcommand{\unit}[1]{\,\mathrm{#1}}
\newcommand{\kelv}{\mbox{}^{\,\circ\!}\mathrm{K}}
\newcommand{\celc}{\mbox{}^{\,\circ\!}\mathrm{C}}
\newcommand{\fahr}{\mbox{}^{\,\circ\!}\mathrm{F}}
\newcommand{\mpa}{\unit{MPa}}
\newcommand{\tauc}{\tau_{\scriptscriptstyle C}}
\newcommand{\taue}{\tau_{\scriptscriptstyle E}}

\newcommand{\vecalpha}{\boldsymbol{\alpha}}
\newcommand{\vecbeta}{\boldsymbol{\beta}}
\newcommand{\vecgamma}{\boldsymbol{\gamma}}
\newcommand{\veceps}{\boldsymbol{\epsilon}}
\newcommand{\veclambda}{\boldsymbol{\lambda}}
\newcommand{\vecsigma}{\boldsymbol{\sigma}}
\newcommand{\vectheta}{\boldsymbol{\theta}}
\newcommand{\vecomega}{\boldsymbol{\omega}}
\newcommand{\vectau}{\boldsymbol{\tau}}
\newcommand{\vecnu}{\boldsymbol{\nu}}
\newcommand{\vecrho}{\boldsymbol{\rho}}
\newcommand{\vecxi}{\boldsymbol{\xi}}
\newcommand{\vecphi}{\boldsymbol{\phi}}
\newcommand{\vecPhi}{\boldsymbol{\Phi}}
\newcommand{\vecpsi}{\boldsymbol{\psi}}
\newcommand{\vecPsi}{\boldsymbol{\Psi}}
\newcommand{\vecDelta}{\boldsymbol{\Delta}}
\newcommand{\vecGamma}{\boldsymbol{\Gamma}}
\newcommand{\vecNabla}{\boldsymbol{\nabla}}
\newcommand{\vecSigma}{\boldsymbol{\Sigma}}

\newcommand{\eff}{ef\!f}
\newcommand{\rf}{re\!f}
\newcommand{\defl}{de\!f\mkern-2mul}
\newcommand{\fc}{\mkern-2muf\mkern-2muc}
\newcommand{\ofs}{o\mkern-2muf\mkern-2mus}
\newcommand{\parll}{\mathbin{\!/\mkern-5mu/\!}}


\title{Programmer guide for CONTACT, Getting started with the source code}
\author{Dr.ir.\ E.A.H.\ Vollebregt}
\cprtext{\copyright\ \vtechcmcc.}
\date{\today}
\reportnumber{23-01, version `trunk'}

\parindent 0mm
\parskip 1ex

\begin{document}
\maketitle

\vspace{2ex} % (i.g.v. titel verdeeld over twee regels)

%\begin{vtlogsheet}
%\vtlogentry{0.1}{EV}{25-10-2021}{Initial version of report}{}
%% \vtfilelocation{CMCC/Verkoop/Strategie}
%\end{vtlogsheet}

\chapter{Introduction}

This document gives a quick introduction to the open-source version of the
CONTACT software. A diverse set of aspects is covered, including some
general programming advice, some background on Fortran, and discussion of
the main aspects of the CONTACT program itself.

{\em This document could be used to start a Wiki, for easier further
elaboration.}

\chapter{Top ten advice for scientific programming}

Here's VORtech's top-10 guidelines for scientific programming. Created in
2010, but still fairly up to date.
\begin{enumerate}
\item\label{itm:clarity} Above all, strive for clarity of the code.
        Complexity is the worst enemy when your programs get bigger.
\item\label{itm:version_mngmt} Use a version management tool.
\item\label{itm:coding_std} Use good coding standards. Uniformity is more
        important than making your parts look nice according to your own
        standards. 
\item\label{itm:automate_testing} Automate testing, so that you can test
        often.
\item\label{itm:know_sensitivity} Get to know the sensitivity of your
        calculations: distinguish noise due to round-off from errors in
        the code.
\item\label{itm:design} Make a design.
\item Develop incrementally.
\item\label{itm:debug_facility} Create visualization and debugging facilities.
\item\label{itm:dont_optimize} Don't optimize until necessary.
\item Use an issue management tool.
\end{enumerate}
Additional practices that are recommended are as follows:
\begin{enumerate}\setcounter{enumi}{10}
\item\label{itm:think_early} Think before you start: what you need, how to
        do it, how to test if it's working. {\em Shift left\/}. The cost of
        errors increases dramatically the longer they are left in the system.
\item\label{itm:lean_on_compiler} Lean on the compiler to detect
        programming errors.
\item\label{itm:build_system} Don't save on the build system. Errors due to
        parts not being recompiled or compiled differently are particularly
        difficult to trace.
\item {\em Boy scout rule}: leave things better than you found them. Polish,
        clean up the parts that you studied. % {\em Continual Improvement\/}.
\end{enumerate}
A nice book with lots of advice from Google: \cite{Winters2020}, and
another one collecting wisdom of many experienced programmers:
\cite{Henney2010}.

\chapter{Programming on CONTACT}

\section{Development environment}

Vollebregt does not use an Integrated Development Environment (IDE) such as
Visual Studio. Instead, he works with an editor (GVIM) and makefiles.
\begin{itemize}
\item CONTACT is programmed mostly using Fortran, with some bits in C and
        the CONTACT GUI programmed in Java.
\item Vollebregt uses Matlab a lot, some scripting in Perl, and some
        programming in Python.
\item The main compiler used is the one from Intel, with the corresponding
        facilities of the Intel Math Kernel Library (MKL).
\item Separate makefiles are maintained for Windows (using {\tt nmake}) and
        Linux (GNU {\tt make}), set up in a generic way, with configuration
        options hidden in include files.
\end{itemize}

\section{Makefiles}

Makefiles are used to automate compilation. This uses a specific (complex)
language of ``targets'' and ``dependencies''. Only those targets are
compiled (anew) that are ``out of date'', using the rules and commands
defined in the makefile.

We use separate makefiles for Windows (using nmake) and Linux (GNU make).
Each makefile can generate multiple outputs for the stand-alone program or
the CONTACT library, with or without OpenMP (multithreading), with
different options for debugging or naming of symbols (with/without
underscores). We also used to create separate versions for 64bit and 32bit
platforms, but the latter are no longer used.

\section{Coding standards}

Coding standards help to make code more uniform and enforce practices to
make the code easier to understand (best practice \ref{itm:coding_std},
page \pageref{itm:coding_std}).
\begin{itemize}
\item The screen width is set to 110 characters. No line should be longer
        than that.
\item Indentation (tabs) goes with 3 characters.
\item Using {\tt implicit none} all of the time (best practice
        \ref{itm:lean_on_compiler}, page \pageref{itm:lean_on_compiler}).
\item Use lots of white space, to facilitate identifying the separate
        constituents of compound statements.
\item Avoid uppercase characters, except for constants.
\item Use underscores in variable names with different parts.
\item Type names start with {\tt t\_}, e.g.\ {\tt t\_grid}, defined in the
        module {\tt m\_grids.f90}. The prefix {\tt m\_} is used for
        modules, {\tt p\_} for pointers.
\item Subroutines specific for a derived type start with the type it is
        used for, e.g.\ {\tt grid\_copy}.
\item Relational operators {\tt .lt.}, {\tt .le.}, {\tt .eq.}, etc.\ are
        used instead of the alternative forms {\tt <=}, {\tt <}, {\tt ==}.
        This is a subjective, personal preference of Vollebregt, imposed on
        others for the uniformity of the code.
\end{itemize}

\section{Conditional compilation}

Conditional compilation is used to make different executables from the same
source code.
\begin{itemize}
\item Platform dependence, e.g.\
{\small\begin{verbatim}
#if defined _WIN32 || defined _WIN64
   integer,          parameter :: platform = plat_win
   character(len=1), parameter :: path_sep = '\\'
#else
   integer,          parameter :: platform = plat_lnx
   character(len=1), parameter :: path_sep = '/'
#endif
\end{verbatim}}
\item Compiling with one or another compiler:
{\small\begin{verbatim}
#ifdef PLATF_ppc
   icount = IARGC()
#else
   icount = command_argument_count()
#endif
\end{verbatim}}
\item Compiling with or without OpenMP (multithreading):
{\small\begin{verbatim}
#ifdef _OPENMP
   iparll = omp_in_parallel()
#else
   iparll = 0
#endif
\end{verbatim}}
\item Compiling with or without certain features:
{\small\begin{verbatim}
#ifdef WITH_MKLFFT
   use mkl_dfti
#endif
\end{verbatim}}
\end{itemize}
Some of the variables used are set automatically by the compiler ({\tt
\_WIN32}, {\tt \_OPENMP}), others are set using compiler options ({\tt
-DWITH\_MKLFFT}).

\section{(Semi-)automated testing}

Scripts are used to run a series of tests, one by one, and pull up a diff
of the new and reference output (best practice \ref{itm:automate_testing},
page \pageref{itm:automate_testing}). Especially {\tt cleanup.pl},
{\tt run\_tests.pl}, and {\tt set\_refout.pl} in the {\tt examples} folder
(programmed in Perl).

{\em It is desired to reduce the number of scripting languages used,
reprogramming Perl and shell scripts in Python.}

We go through the differences manually, judging whether these are
acceptable or not. We often find slight differences in the numerical
outputs due to round-off errors. Differences between Windows and Linux or
just between two runs of the compiler after a code modification. This
round-off error cannot be circumvented. It takes some training to judge
quickly where this occurs (best practice \ref{itm:know_sensitivity}, page
\pageref{itm:know_sensitivity}).

{\em It is desired to automate testing.}

\section{Version identification}

CONTACT prints the revision number from Subversion in its startup message:
\begin{alltt}
   ------------------------------------------------------------------------
   |  CONTACT - detailed investigation of 3D frictional contact problems  |
   |                                                                      |
   |  Version: \red{trunk(64), $Revision: 2116 $, $Date:: 2022-01-14$}          |
   |  Copyright Vtech CMCC, all rights reserved.                          |
   |  \red{(Subversion revision 2116M, 14-Jan-2022 17:00)}                      |
   ------------------------------------------------------------------------
\end{alltt}
This facilitates the comparison of different outputs, e.g.\ reference
output from one version and new output from a modified version. If
differences need be explained, we can then use bisection to identify the
version where those differences can be seen for the first time.

{\em This needs to be extended for the open source version using Git for 
version control.}

\section{Modern Fortran}

The first version of CONTACT was developed using Fortran IV. Then came
Fortran77 and Fortran90. Today we use some features of Fortran 2003 or
2008. 

A nice feature is to have multiple subroutines used under a generic name.
An example is the {\tt write\_log} function from {\tt m\_print\_output}.
This can be called with one argument, a literal character string, or two
arguments, an array of strings and the number of lines it contains:
{\small\begin{verbatim}
   public write_log
   interface write_log
      module procedure write_log_str
      module procedure write_log_msg
   end interface write_log

   subroutine write_log_str(str)
      !--purpose: Write a single string to the log output streams
      implicit none
      !--subroutine arguments
      character(len=*), intent(in) :: str
   end subroutine write_log_str

   subroutine write_log_msg(nlines, msg)
      !--purpose: Write an array of strings to the log output streams
      implicit none
      !--subroutine arguments
      integer,          intent(in) :: nlines
      character(len=*), intent(in) :: msg(nlines)
   end subroutine write_log_msg
\end{verbatim}}
If the compiler then encounters a call like {\tt write\_log('some
message')}, it will try to match the actual arguments (one literal string)
with each of the options. The example matches with {\tt write\_\-log\_\-str},
which is then the actual subroutine that will be used.

See also operator overloading in Section \ref{sec:m_markers}.

\section{Pointers and associate}

Pointers are slightly different in Fortran than in other languages like C
or C++. Whereas assignment `{\tt =}' changes the pointer value (memory
address) in C/C++, in Fortran this changes the variable that is pointed to
(memory contents). In Fortran, the assignment `{\tt =>}' is used to change
the pointer value (memory address).

Pointers in Fortran behave much like nick-names for the names of other
variables, e.g.\
{\small\begin{verbatim}
   type(t_profile), pointer :: my_rail

   my_rail => wtd%trk%rai
      ...
\end{verbatim}}
This allows {\tt my\_rail} to be used as shorthand for {\tt wtd\%trk\%rai}
and behave like any other variable of type {\tt t\_profile}. The associate
clause achieves mostly the same without the variable declaration:
{\small\begin{verbatim}
   associate( my_rail => wtd%trk%rai )
      ...
   end associate
\end{verbatim}}
Pointers are used only minimally in CONTACT, because of the risk they have
for allocating memory and forgetting about deallocation. Associate is being
used more and more throughout the code.

\section{Lazy implementation}
\label{sec:lazy_implem}

The modules/objects used in CONTACT are expanded using a strategy called
``lazy implementation''. Only those parts are implemented that are actually
needed or foreseen to be needed.

An example concerns the construction of a rotation matrix from Euler angles
({\tt t\_rotmat} in {\tt m\_\-mar\-kers}). Four functions are defined for this:
{\small\begin{verbatim}
   function rotmat_pure_roll
   function rotmat_pure_yaw
   function rotmat_pure_pitch
   function rotmat_roll_yaw
\end{verbatim}}
In principle, additional variants could be defined for {\tt roll\_pitch}, 
{\tt yaw\_pitch}, and {\tt roll\_\-yaw\_\-pitch}. There has been no need for
these functions (yet), so they have not (yet) been implemented.

A typical error message in a generic function would be ``this functionality
(variant, combination) is not yet supported''.

Adding all variants that logically fit together creates ballast: additional
time for compilation, and additional maintenance work.

\chapter{Internal structure of CONTACT}

\section{Stacking of modules}

\begin{figure}[p]
\centering
\psfig{figure=fig/contact_sw_modules2,width=5.0in,trim=5 155 20 15,clip=}
\caption{\em Stacking of the Fortran90 modules used in CONTACT.}
\label{fig:arch_modules2}
\end{figure}

Figure \ref{fig:arch_modules2} shows how different modules in CONTACT build
on each other (best practice \ref{itm:design}, page \pageref{itm:design}). 
\begin{itemize}
\item The makefiles are at the bottom, together with facilities provided
        by SoftwareKey (licensing mechanism) and Intel (MKL). 
\item Right above this we find facilities for printing, as needed to
        integrate the CONTACT output in the output streams of software
        packages like SIMPACK or RecurDyn.
\item Going upwards we find generic facilities (timing, input, licensing,
        BLAS), quite generic facilities oriented towards CONTACT (markers,
        grids, grid functions, interpolation), and more specific facilities
        (friction, influence coefficients, w/r profiles).
\item Next comes the aggregation in {\tt m\_hierarch\_data}. This provides
        the data-type {\tt t\_probdata} that groups all the elements used
        for a single ``contact problem'' in module 3.
\item The layer above {\tt m\_hierarch\_data} shows the components used
        in the stand-alone program in module 3: I/O, preparations, solving,
        and specific computations.
\item The layer {\tt m\_wrprof\_data} provides the data-type {\tt
        t\_ws\_track} that groups all the elements used for a single
        ``w/r contact problem'' in module 1. This uses many aspects of
        {\tt m\_\-hierarch\_\-data}, and adds specific elements for the analysis
        of wheel/rail contact analysis.
\item The layer above {\tt m\_wrprof\_data} shows the additional components
        used for solving w/r contact problems.
\item The top layer adds the main program and driver routines, and the
        additional data and wrapper functions for the CONTACT library
        version.
\end{itemize}


\section{Scope of modules}

Modules are used to group functionality that belongs together and separate
from other functionality that is less related. ``High cohesion, loose
coupling'' (best practice \ref{itm:design}, page \pageref{itm:design}).

Some modules are centered around a derived type. An example of this is {\tt
m\_friclaw}, that hides the many facets of different friction laws from the
rest of the program. This module includes the input from the {\tt
inp}-file and output to the {\tt out}-file specific to friction. This uses
ideas from object orientation.

Other modules just combine related computations. Examples are {\tt
m\_solvpt} for different solution algorithms for the tangential problem,
and {\tt m\_sdis} for different preparations for the actual solver
routines.

\section{Hierarchical data-structure}

The implementation of CONTACT revolves around ``hierarchical
data-structures'' as defined in {\tt m\_\-hie\-rarch\_\-data.f90} and {\tt
m\_wrprof\_data.f90}.

The main object in contact is a ``contact problem''. In module 1, this is
encoded in the type {\tt t\_ws\_\-track}, in module 3 it is {\tt
t\_probdata}. These types group together all the inputs, configuration,
intermediate results, and outputs of a single contact problem. 

In module 3 these data comprise
{\small\begin{verbatim}
   ! aggregate of all data for a CONTACT calculation in module 3:

   type :: t_probdata
      type(t_metadata) :: meta   ! meta-data describing the calculation
      type(t_scaling)  :: scl    ! scaling factors for the CONTACT library
      type(t_ic)       :: ic     ! integer control digits
      type(t_material) :: mater  ! material-description of the bodies
      type(t_potcon)   :: potcon ! description of potential contact area
      type(t_grid)     :: cgrid  ! main discretisation grid for CONTACT
      type(t_hertz)    :: hertz  ! Hertzian problem-description
      type(t_geomet)   :: geom   ! geometry-description of the bodies
      type(t_friclaw)  :: fric   ! input parameters of friction law used
      type(t_kincns)   :: kin    ! kinematic description of the problem
      type(t_influe)   :: influ  ! combined influence coefficients
      type(t_solvers)  :: solv   ! variables related to solution algorithms
      type(t_output)   :: outpt1 ! solution and derived quantities
      type(t_subsurf)  :: subs   ! data of subsurface stress calculation
   end type t_probdata
\end{verbatim}}
A number of these items are re-used in the contact problem central to module
1. Additional data used there are
{\small\begin{verbatim}
   ! aggregate data for the half wheelset on track/roller combination:

   type :: t_ws_track
      type(t_discret)  :: discr   ! parameters used for potential contact
                                  !     areas and discretization
      type(t_vec)      :: ftrk, ttrk, fws, tws, xavg, tavg
      real(kind=8)     :: dfz_dzws ! sensitivity of fz_tr to z_ws

      type(t_wheelset) :: ws      ! wheel-set data, including wheel profile
      type(t_trackdata):: trk     ! track/roller data, including rail profile
      integer          :: numcps  ! number of contact problems
      type(p_cpatch)   :: allcps(MAX_NUM_CPS) ! pointers to the data for
                                  !     all contact problems of ws on trk
   end type t_ws_track
\end{verbatim}}
Module 1 uses an array of ``contact patches'', where each contact patch
eventually contains the data for a ``contact problem'' as defined in module
3.

A central object in CONTACT is a ``contact patch''. This is defined
primarily using a ``potential contact area'' and a corresponding ``CONTACT
grid'' of $m_x\cdot m_y$ rectangular elements of size $\delta
x\times\delta y$. Module 1 identifies and defines contact patches, module 3
performs the solution.

\section{Module m\_markers, operator overloading}
\label{sec:m_markers}

Module {\tt m\_markers} provides the derived types (classes?) {\tt t\_vec},
{\tt t\_rotmat}, and {\tt t\_marker}. The main ideas behind these are
explained in the paper \cite{Vollebregt2020b-wrgeom}. 

A fourth type that is defined is {\tt t\_coordsys}. The idea behind this is
to provide automated checking of compatibility of markers used in
computations. This hasn't been worked out.

The module uses operator overloading to define standard operations like
addition of vectors, multiplication of a rotation matrix times vector, or
the product of two rotation matrices, e.g.
{\small\begin{verbatim}
public operator (+)
interface operator (+)
   procedure vec_add
end interface operator (+)

function vec_add(v1, v2)
   type(t_vec), intent(in) :: v1, v2  !--inputs
   type(t_vec)             :: vec_add !--result value
end function vec_add
\end{verbatim}}
If the compiler then encounters the addition of two vectors, it will call
the function as follows:
{\small\begin{verbatim}
   ftot = vec( 0d0, 0d0, 0d0 )
   do icp = 1, numcps
      cp   => wtd%allcps(icp)%cp
      ftot = ftot + cp%ftrk   ! --> ftot = vec_add(ftot, cp%ftrk)
   enddo
\end{verbatim}}

\section{Module m\_ptrarray}

Module {\tt m\_ptrarray} provides a number of generic facilities,
esp.\ {\tt reallocate\_arr}, that permit changing size of arrays during
program execution.

\section{Module m\_grids}

Module {\tt m\_grids} defines the derived type {\tt t\_grid} with
corresponding functions. The main ideas behind this are presented in
\cite{Vollebregt2020b-wrgeom}.

There are actually three different types of grids that are squeezed into a
single derived type {\tt t\_grid}: curves, used to represent profiles,
curvilinear grids, used to represent wheel and rail surfaces, and uniform
grids, used for the potential contact area. Benefits of combining into one
type are that this avoids repetition (translating a profile, translating a
surface), and that this facilitates conversion from one type to another
(profile to surface).

The module does not aim to provide a fully generic grid data-type. Only
those grids are supported that are needed by CONTACT. The operations
provided came about by the process of ``lazy implementation'' (\S 
\ref{sec:lazy_implem}). More functions can be added as needed in CONTACT.

For profiles, the grid used may be complemented with a spline
approximation (\S \ref{sec:m_interp}).

\section{Interpolations: m\_interp, m\_spline\_*}
\label{sec:m_interp}

Interpolations are essential to CONTACT's analysis of the wheel/rail
contact geometry problem. These are used in two different forms: 
\begin{enumerate}
\item subroutines for 2d bilinear interpolation on curvilinear grids,
        provided in {\tt m\_interp},
\item subroutines related to splines, for profile smoothing and
        interpolation.
\end{enumerate}
A lot of code has been developed for working with splines. This started
with ``PP-splines'' to represent plane curves (profiles), adding smoothing,
adding ``B-spline curves'', and finally adding ``B-spline surfaces''.
These are typically stored together with a grid definition. The original
PP-form is implemented in {\tt m\_spline\_def}, {\tt make}, {\tt get};
extensions for 1D/2D B-splines are implemented in {\tt m\_bspline\_def},
{\tt make}, {\tt get}.

The main ideas related to splines are presented in
\cite{Vollebregt2020b-wrgeom}. The arrays used are stored along with the
original profile data in the {\tt t\_grid} data structure. Crucial
subroutines are {\tt make\_smoothing\_spline\_\-section}, {\tt
grid\_spline\_eval}, and {\tt solve\_cubic\_segm}. Many other routines are
built on top of these cornerstones.

{\em The functionality for splines is still under construction. Needs
completion of the functionality and extension of the documentation.}

The main work-horse for 2D interpolation is subroutine {\tt
interp\_wgt\_surf2unif}, establishing the mapping between a curved surface
and a uniform mesh in an efficient way. A number of variants are built on
top of this, for interpolation of scalar data or 3-vectors, packed in an
array, surface (grid), or grid-function. 

\bibliographystyle{unsrt}
\bibliography{ug_contact}

\end{document}
